%{\textbf{Goal of the proposal to achieve, its focus and why...}}\\

In this section, we address the goal of this proposal to achieve by focusing on concept of background and lessons from
previous cases.
We define that background is the event in which experiments are affected by unwanted particles entering from the machine.
Such background can intervene the machine regular operation and even worse have a potential damage that ends up machine
failure, including Crab cavity failure and asynchronous beam dump.
In general, machine background should be well understood not only in advance of the machine operation but also optimal
design of detectors as well as shielding. The lesson from the HERA II commissioning after the upgrade, very severe
backgrounds were observed, and the machine had to be shut down for several months. While studies of the backgrounds
were performed and solutions to mitigate the backgrounds were installed. Their major source of background comes from
large proton beam-gas interactions and increased synchrotron radiation. Since a proposed Electron-Ion-Collider(EIC)
is similar machine as HERA (electron-proton-collider) although there are many different features, their major background
sources could be same issue.


Recently, LHC had dedicated tests during 2015-2016 to understand machine-induced background through studying in terms of
a loss map for beam-halo, pressure bump for local beam-gas interaction. In particular, it turned out that beam-gas interaction
is the most dominant background by high luminosity experiment analysis results\cite{Bruce:2016}.
In reality, far advance from detail machine and detector design, there are crucial questions related to machine-induced background
should be addressed:
\begin{itemize}
\item[(1)] Can we understand better at what level background become problematic ?
\item[(2)] What can we demonstrate for such background ? How we can evaluate ?
\item[(3)] What can be done to improve safety for design/operation ?
\end{itemize}


Our proposed study will focus on identfying/estimating machine-induced backgrounds to the detectors, giving crucial information
to all groups designing detector systems on the radiation environment in which the devices will operate. Techniques and tools in this study
will commonly be applicable for general EIC design.
In simplification, the goals of this proposal is following:
\begin{itemize}
\item[(1)] Identify the background sources for given baseline design parameters, in particular JLEIC
\item[(2)] Quantify the background level through GEANT4 Monte Calro simulation with beam parameters, vacuum level, equipment material, etc
\item[(3)] Suggest or provide possible solution or modification for problematic radiation level
\end{itemize}

First of all, it is very important to identify what is the background source that causes severe problem during beam operation
or detection capability. For example, synchrotron radiation is very critical source for tracking device near the interaction
point. Hence masking or leveling down this radiation is key to improve the tracking efficiency. Secondly, once we know source
of background, we should evaluate how much background level will be expected at given operational parameters. By doing this
study, it will immediately provide an idea or solution how we can improve or optimize the run configuration or detector design
or shielding. More detail technical approach will be discussed in section 4. We will study together with experts from various fields.


%{\textbf{Aspects of common interest bewteen eRHIC and JLEIC}} \\
We also emphasize the aspects of common interest between eRHIC and JLEIC to maximize the synergy.
In this proposal, we would like to study various source of machine-induced backgrounds, such as Synchrotron radiation,
 Beam-gas interactions, Beam halo, Beam losses, Neutron flux and Radiation from elastic scattering. Such backgrounds
 are major and common concerns independent of machine. However, they are not always clearly identified as separated
 effects.  We hence propose a study for any correlation among these sources with respect to accelerator machine (beam parameters, final focus magnetic fields, etc) and
 detector design (geometry, material, shielding, etc) through sophisticated Monte Calro simulation based on GEANT4 with various event generators.
Therefore this will be a very important study for both eRHIC and JLEIC to understand and evaluate all these sources
with respect to design parameters and operational configurations.
Detailed simulation studies need to be performed for the relevant background sources.
A subset of sources of background we will study are itemized in Section~2.
We have a great test bechmark by demonstrating HERA background rate through GEANT4 simualtion. A detail achievement has been disccussed in Appendix.A.
Such simulation results will be directly input to improve experimental detector response.


