Several project deliverables will be achieved with the completion of this study.  Specifically:
\begin{itemize}[label=\textbullet]
	
		\litem{} Develop simulation code and CAD model for most current interaction region design, especially with regard to new beam pipe design and variations, which is necessary for all future EIC physics simulations. 
	
		\litem{}Develop and validate new and existing simulation packages to model synchrotron radiation and examine path of radiation, shielding design, collimation schemes, and provide recommendations for layout changes if necessary.
		
		\litem{} Use simulation of synchrotron radiation to quantify and track this background in detectors.  Determine the amount and locations of energy deposited and evaluate any impact to  design.
		
		\litem{} Compare background from synchrotron radiation to the rate of physics events in a study of signal to background. 
		
		\litem{} Perform detailed study of vacuum pressure distribution in and around the IR.  This is necessary to achieve an accurate estimate of beam-gas event rates in the detectors.  Furthermore, this study underlies vacuum pump placement and begins an iterative design process in collaboration with the machine design group, especially with regards to beam pipe design.
		
		\litem{} Implement JLEIC configuration and event generator with existing, benchmarked packages to model beam-gas interactions.
		
		\litem{} Quantify and track background from beam-gas interactions to determine impact on beam and physics measurements.
		
		\litem{} Quantify levels of neutron flux by collaborating with Radiation Control division to model experimental area and determine its impact. 
		
		\litem{} Optimize placement of auxiliary detectors for reducing background: for instance, positioning the high luminosity monitor at ion exit arc to reduce hadronic and synchrotron backgrounds.)
					
		\litem{} Provide feedback to the machine design group in an iterative process to help guide the design optimized for physics.
		
		\litem{} Overall, model and quantify background radiative load as a function of angle to help guide detector technology design choices and optimization of placement.  This will form the basis for not only the background studies but future physics simulations.  
	
	\end{itemize}