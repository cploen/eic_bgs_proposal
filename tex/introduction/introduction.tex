
The machine design directly impacts the quantity and type of radiation reaching detectors in the interaction region, which influences the both the physics program and the materials which can be used in the detectors.  Background radiation also greatly affects the systematic uncertainty.  It is important to minimize and fully understand the systematic uncertainty, as it will dominate the high-precision physics measurements at high luminosity.  Specifically, background radiation is influenced by the arrangement of magnet lattice which guides the beam, bunch spacing, beam current, and beam optics, which are being decided now.  Therefore it is critical to perform a thorough study of the type and dose of machine-induced background now, while decisions regarding the interaction region layout are being made.  This insight will enable informed decisions which minimize and mitigate sources of background at the early design phase and to inform detector placement and technology choices as well.  As such, close collaboration between the accelerator and physics divisions will be necessary to inform the design process and maximize the potential of the physics program at the EIC.

Drawing upon the extensive experience of earlier facilities, especially the previous electron-proton collider at HERA at DESY, we are further motivated to perform a detailed study of the backgorund in the IR.  In many ways, the EIC is an extension of the previous collider, which strove to push the limits of luminosity upwards by a factor of seven following the installation and commissioning of the upgrade during 2001/2002.  However, the upgraded machine lattice and beam optics generated severe levels of background in the interaction region.  The primary sources of background were direct synchrotron radiation and beam gas scattering due to vacuum degradation.  After a several-month shutdown, detailed background studies were performed and solutions to mitigate the background were installed.  Ultimately the problems were addressed and the HERA-II upgrade was a success, increasing luminosity by a factor of five; however, this experience underscores the need to complete detailed studies during the design phase as a close collaboration between both physics and accelerator design experts, rather than after the commissioning is complete.  The HERA-II machine shutdown, delays, and post-design changes lend gravitas to our proposed study and strengthens the case for funding.

In summary, we propose a study to determine the quantity, type, and distribution of machine-induced background in the interaction region.  These findings will provide crucial insight to both accelerator and physics divisions, informing choices for interaction region design and detector technology and placement.  Comprehensive knowledge about the radiative environment of the interaction region is essential to guide the current and future design phases.

