
Detailed simulations must be performed for relevant background sources.  Sources of background observed at other facilities are listed below, followed by relevant information and its potential impact on detectors.
\begin{itemize}
	\item Synchrotron radiation
	\item Beam-gas interactions
	\item Beam halo
	\item Beam loss
	\item Neutron flux
	\item Radiation from elastic scattering
\end{itemize}

\subsection{Synchrotron radiation}

Synchrotron radiation can significantly impact physics measurements in both main and auxiliary detectors.  It is primarily generated as the electron beam bends through the machine lattice magnets.  In addition to impacting the detector systems, synchrotron radiation can have a deleterious effect on the vacuum quality by heating residual beam gas or causing enhanced desorption of gas from vacuum chamber walls.  This, in turn, causes beam loss due to interactions with the ion beam.  Furthermore, excessive heating due to synchrotron radiation could damage flanges, gaskets, and certain types of detector technology.

Current JLEIC design attempts to reduce the amount of synchrotron radiation generated and lessen its effects on the ion beam and surrounding detectors.  The long, straight section of electron beam line pipe prevents much contribution from the magnet lattice, and the large beam crossing angle mitigates the effect of synchrotron radiation on vacuum quality in the IR. [reference S. Sosa crab X and/or F. Lin MEIC Background 1st look]
%Elke disusses placement of the lumi monitor, need for bgs to provide input to this particular detector.  Do we have a specific detector to discuss at this point.
%Elke mentions the qualifications of current team members and experience with study so far.
%Elke discusses particular effects of SR on HERA upgrade and increasing thickness of SR shield at ZEUSS lumi monitor.

\subsection{Beam gas interactions}

Beam-gas interactions occur when proton or ion beam particles collide with residual beam gas.  That is, these interactions are fixed-target p+A or A+A collisions.  The problematic background experienced after the HERA II upgrade was predominantly due to such events.  Design choices in the upgraded IR, such as the zero angle crossing and long section of shared beam pipe, exacerbated the beam-gas problem.  Synchrotron radiation heated the beam pipe, which released residual gas particles from the beam pipe walls and degraded the vacuum.  This increased the number of beam-gas interactions at the IR and near the detectors.  The large background rate was mitigated by “baking out” the beam pipe gas by running high proton beam currents for an extended time and by regularly warming up the final focus quadrupoles to remove frozen gas.  

While the JLEIC design incorporates a large crossing angle and limited shared beam pipe region to reduce the effect of beam-gas interactions, the baseline beam currents are significantly higher than those of HERA II.  Thus, detailed simulations to compare to the HERA II experience and determine a working range of vacuum quality are proposed. 

\subsection{Beam halo}

Particles formed from elastic collisions of both electron and proton beams with residual gas or beam-beam interactions can form a halo distribution around the beam.  Often the result is an on-momentum electron or ion with large scattering angle. These particles can then generate additional background by interacting with the beam pipe and can impact the stability of the beam.  Beam halo must be studied in order to determine whether “scraping” the halo with collimators is required, as well as proper placement of those collimators.
%[reference: BNL EIC SLIDE]
%See HERA: Sources and Cures (p. 6) for brief discussion of collimating beam halo
%and issues from external distortion i.e. Cultural noise. 

\subsection{Beam loss}

Unexpected beam loss can significantly damage detector systems unless the proper controls are in place.  Magnet fires, power supply failure, or other situations can cause the interruption of beam unless machine protection systems are in place.  Failsafe measures need to be studied and can influence the optimal placement of detectors and protections in the interaction region.

\subsection{Neutron flux}

Neutrons with energies around a few hundred keV can be detrimental to detector components. For instance, silicon photo-multiplier tubes are especially vulnerable.  A quantitative estimate of the neutron flux is needed for detector development and placement.  To achieve this, modeling the experimental hall and collaborating with the Radiation Control Group is proposed as an extension to this study.

\subsection{Radiation from elastic scattering}
%Here Elke discusses Bethe-Heitler process and "signal" for lumi monitor 
%is effectively noise for low q^2 tagger.  This is great example.  Do we have one similar or same?


