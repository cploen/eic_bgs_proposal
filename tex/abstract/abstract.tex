%first line taken from BNL proposal
This proposal seeks funding to research related to the background sources in the experimental detectors due to the operation of the JLEIC accelerator.


Detectors and machine operation been seriously impacted by machine-induced background at earlier facilities.  Experience has shown that synchrotron radiation and beam interactions with residual beamline gas are major sources of background that require mitigation.  However, other sources must be investigated and their impact assessed.

The type and quantity of background signal impacts the technologies appropriate for the central and auxiliary detectors.

Thus, this study aims to identify the dominant background sources, quantify their impacts on the detectors and physics, and explore mitigation options as needed.

These studies are essential and should be developed alongside design of the interaction region, since the backgrounds are dependent on details of the machine lattice and beam optics.   

Knowledge gained will provide critical information to the machine and detector designs.  Collaboration between the accelerator and physics groups is necessary and will support an iterative design process that maximizes the capabilities of the physics program at the Jefferson Lab EIC.

