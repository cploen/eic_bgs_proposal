%first line taken from BNL proposal
In this proposal, we request funding to perform simulations in great detail of the EIC machine related backgrounds and, by quantitative procedure, evaluate the background radiation reaching the experimental detectors. The detectors must be sufficiently well protected to prevent both excessive component occupancies and deterioration from radiation damage. Experience has shown that synchrotron radiation and beam interactions with residual beamline gas are major sources of background that require mitigation. However, other sources must be investigated and their impacts assessed. The type and quantity of background signal impacts the technology choices for the central and auxiliary detectors. There is an effort at Brookhaven National Lab to study beam gas interactions; however, it is primarily focused on the eRHIC design. We propose to independently verify the results of that study and apply the concepts developed at BNL to the JLEIC configuration. Moreover, we would like to advance these studies further by designing a detector beam pipe, considering the impact of the crab crossing, and modeling an extensive section of the accelerator. Thus, this study aims to identify the dominant background sources, quantify their impacts on the detectors and physics, and explore mitigation options as needed. These studies are essential and should be developed alongside the interaction region design, since the backgrounds are dependent on details of the machine lattice and beam optics. Knowledge gained will provide critical information to the machine and detector designs at both JLEIC and eRHIC. Collaboration between the accelerator and physics groups from both laboratories is necessary and will support an iterative design process that maximizes the capabilities of the physics program at an EIC. 



%add hadronic and leptonic background fluctuations due to crab crossing bunch tilt in horizontal plane.

%add development of beam pipe shape which optimizes acceptance and iminizes interaction with beam pipe particles.

%add synchrotron radiation contributions from electron beam interacting with residual beam gas and evaluation of small chicane to minimize this source of background.

% add first look at vacuum system design by determining viability of pump placement options and baseline vacuum level achievable with minimal scheme.

%add checking BNLs results with independent approach (benchmarking their results?)

