This study seeks to develop a comprehensive understanding of the sources and quantities of background generated around the interaction region (IR) during the operation of the JLEIC accelerator. Earlier facilities have been greatly impacted primarily by synchrotron radiation and ion beam-gas interactions.  However, other sources of background must be investigated to determine their impact on the IR.  Technology choices, placement, and design parameters for the main detector and auxiliary detectors depend on a thorough understanding of sources and quantities of background.  Therefore, this study seeks to develop simulations in order to identify the types of background generated near the IR, quantify their count rates and energy depositions at points of interest in the region, and explore mitigation options as needed.  Furthermore, conducting this study in parallel with the interaction region design permits an opportunity to optimize optics and the machine lattice while minimizing background. Knowledge about the sources and magnitudes of background are necessary to guide both the detector and machine designs in a way to fully realize the experimental capabilities of the JLEIC accelerator. 